\documentclass{article}
\usepackage{hyperref}
\usepackage{enumitem}
\title{Vulkan Notes}
\author{Anton Khabbaz}

\begin{document}
\maketitle
\section{GLFW Commands}
\begin{description}
\item[ glfwInit()] initialize window
\item[glfwWindowHint()] set properties of the window Some are:
	\begin{itemize}
		\item \verb|GLFW_CLIENT_API, GLFW_NO_API| do not create an
OpenGL context.
		\item \verb|GLFW_RESIZABLE, GLFW_FALSE| do not allow resizing of
windows.
	\end{itemize}
\item [glfwCreateWindow()] with argument 
 (800, 600, "Vulkan", nullptr, nullptr) creates a window
and return a pointer to it.  First two must be width and height, next is name.
The fourth parameter allows you to optionally specify a monitor to open the
window on and the last parameter is only relevant to OpenGL.	
\end{description}
\section{Initialization Commands and Structures}
This follows from the
\href{https://vulkan-tutorial.com/en/Drawing\_a\_triangle/Setup/Instance}{Triangle}
and the initial chapter of VulkanProgramming Guide(OpenGL).  

To create an instance of a structure in Vulkan one uses
\begin{verbatim}
VkResult vKCreateInstance( 
	const vkInstanceCreateInfo*        pcreateInfo,
	const vKAllocationCallbacks*       pAllocator,
	VkInstance*			   pInstance);
\end{verbatim}

The structure inputted is a constant so that is only a source of information, and the
instance returns the resulting instance data.  The \emph{instance} holds all the
tracked states of the application.

The OpenGL programming guide uses the following convention:
\begin{description}
	\item[pInstance]  pointer to instance
	\item[ppEnabledExtensionNames] pointer to pointer to type in this case a
		const char.
\end{description}

\end{document}
