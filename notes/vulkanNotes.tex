\documentclass{article}
\usepackage{hyperref}
\usepackage{enumitem}
\title{Vulkan Notes}
\author{Anton Khabbaz}

\begin{document}
\maketitle
\section{GLFW Commands}
\begin{description}
\item[ glfwInit()] initialize window
\item[glfwWindowHint()] set properties of the window Some are:
	\begin{itemize}
		\item \verb|GLFW_CLIENT_API, GLFW_NO_API| do not create an
OpenGL context.
		\item \verb|GLFW_RESIZABLE, GLFW_FALSE| do not allow resizing of
windows.
	\end{itemize}
\item [glfwCreateWindow()] with argument 
 (800, 600, "Vulkan", nullptr, nullptr) creates a window
and return a pointer to it.  First two must be width and height, next is name.
The fourth parameter allows you to optionally specify a monitor to open the
window on and the last parameter is only relevant to OpenGL.	
\end{description}
\section{Coding Conventions}
\begin{description}
\item[vk] functions
\item[Vk] enumerations and structs
\item[VK\_] enumeration values
\end{description}
\section{Initialization Commands and Structures}
This follows from the
\href{https://vulkan-tutorial.com/en/Drawing\_a\_triangle/Setup/Instance}{Triangle}
and the initial chapter of VulkanProgramming Guide(OpenGL).  
o
To create an instance of a structure in Vulkan one uses
\begin{verbatim}
VkResult vKCreateInstance( 
	const vkInstanceCreateInfo*        pcreateInfo,
	const vKAllocationCallbacks*       pAllocator,
	VkInstance*			   pInstance);
\end{verbatim}

The structure inputted is a constant so that is only a source of information, and the
instance returns the resulting instance data.  The \emph{instance} holds all the
tracked states of the application.

The OpenGL programming guide uses the following convention:
\begin{description}
	\item[pInstance]  pointer to instance
	\item[ppEnabledExtensionNames] pointer to pointer to type in this case a
		const char.
\end{description}
\section{Vulkan Commands}

\begin{description}
    \item[vkEnumerateInstanceLayerProperties](\&layer\-Count,
available\-Layers\-.data())
    will check what layers are available and return in a vector. The struct is
    \verb|VkLayerProperties|.  The layers need to be enabled.  \begin{verbatim}
``VK\_LAYER\_KHRONOS\_validation''
\end{verbatim}
\item[vkEnumerateInstanceExtensionProperties] 
	(nullptr, &extensionCount, 
			extensions.data());
 uses a struct \verb|VkExtensionProperties|. This returns all the extensions
available to vulkan.

\end{description}
\section{Enhancing Vulkan}
\begin{description}
\item[Layers] are for
debugging, parameter validation and logging and printing. Hello triangle uses
one layer that starts many of them called  

\item[Extensions]
Adds features to vulkan by incorporating functions and software into the api.
There are \emph{instance} extensions that enhance all of Vulkan, and
\emph{Device} extensions that are only available on certain devices. These
define new functions. GLFW needs specific extensions given by:
\begin{verbatim}
 glfwExtensions = glfwGetRequiredInstanceExtensions
				(&glfwExtensionCount);
\end{verbatim}
\end{description}
\section{debug Callbacks}
Here is a summary of how you show error messages from Vulkan as a result.  The
mechanism uses a callback that is a static function of the correct signature
which vulkan calls whenever there is an event that  was requested. 

First you populate a structure with vital information for the callback:
\begin{verbatim}
VkDebugUtilsMessengerCreateInfoEXT& createInfo;
    //VK_STRUCTURE_TYPE is a signature for sType field.
	createInfo.sType  = VK_STRUCTURE_TYPE_DEBUG_UTILS_MESSENGER_CREATE_INFO_EXT;
	createInfo.messageSeverity = VK_DEBUG_UTILS_MESSAGE_SEVERITY_VERBOSE_BIT_EXT | 
				     VK_DEBUG_UTILS_MESSAGE_SEVERITY_WARNING_BIT_EXT | 
                                     VK_DEBUG_UTILS_MESSAGE_SEVERITY_ERROR_BIT_EXT;
        createInfo.messageType = VK_DEBUG_UTILS_MESSAGE_TYPE_GENERAL_BIT_EXT | 
			         VK_DEBUG_UTILS_MESSAGE_TYPE_VALIDATION_BIT_EXT | 
                                 VK_DEBUG_UTILS_MESSAGE_TYPE_PERFORMANCE_BIT_EXT;
        createInfo.pfnUserCallback = debugCallback;
        createInfo.pUserData = nullptr; // Optional   
\end{verbatim}
The messageSeverity field says what level of error should trigger the callback.
The \verb|...VERBOSE_BIT_EXT| means show diagnostic messages. Each message is
\begin{verbatim}
    VK_DEBUG_UTILS_MESSAGE_SEVERITY_VERBOSE_BIT_EXT: Diagnostic message
    VK_DEBUG_UTILS_MESSAGE_SEVERITY_INFO_BIT_EXT: 
		Informational message like the creation of a resource
    VK_DEBUG_UTILS_MESSAGE_SEVERITY_WARNING_BIT_EXT: 
		Message about behavior that is not necessarily an error, 
		but very likely a bug in your application
    VK_DEBUG_UTILS_MESSAGE_SEVERITY_ERROR_BIT_EXT: Message about behavior 
		that is invalid and may cause crashes
\end{verbatim}

Notice there is a callback function to register.

The actual callback has this signature:
\begin{verbatim}

// debugCallback with a standard signature.
//VKAPI_ATTR VKAPI_CALL specify the argument order for the function, they say so
//this has the correct signature so Vulkan can call it.
//
// first parameter looks like an enum describing error levels. The higher they
// are the more severe.
static VKAPI_ATTR VkBool32 VKAPI_CALL debugCallback(
    VkDebugUtilsMessageSeverityFlagBitsEXT messageSeverity,
    VkDebugUtilsMessageTypeFlagsEXT messageType,
    const VkDebugUtilsMessengerCallbackDataEXT* pCallbackData,
    void* pUserData) 
\end{verbatim}
This will get called whenever the message Severity is one of the selected types.

First the callback function must be registered:
\begin{verbatim}

VkResult CreateDebugUtilsMessengerEXT(VkInstance instance, const VkDebugUtilsMessengerCreateInfoEXT* pCreateInfo, 
    const VkAllocationCallbacks* pAllocator, VkDebugUtilsMessengerEXT* pDebugMessenger) {
    
//vkGetInstanceProcAddr gets a pointer to any function in Vulkan.  The
//parentheses cast that pointer which is of type  PFN_vkVoidFunction to the type
//of the function you want.  Somehow auto figures this out.

	auto func = (PFN_vkCreateDebugUtilsMessengerEXT) vkGetInstanceProcAddr(instance, 
			"vkCreateDebugUtilsMessengerEXT");
    if (func != nullptr) {
        return func(instance, pCreateInfo, pAllocator, pDebugMessenger);
\end{verbatim}
The last func does the creation.  It needs the instance, the createInfo
structure that specifies everything about the callback, and returns a
pDebugMessenger pointer that should be saved.  After this I think the callback
works.  

In the hello Triangle code, pDebugMessenger is saved in the class.  There is
also a pointer to the createInfo structure saved in pNext in the instance, but
that does not seem needed at all at this stage.  Finally, you need to destroy
the callback when its all through with another function call.


\end{document}
